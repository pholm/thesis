% Abstract in English
% ------------------------------------------------------------------
\thesisabstract{english}{
\textit{Introduction}: Effective team collaboration is essential for the success of software development projects. One key aspect of collaboration is ways of working, defined as explicit and implicit rules and norms that guide team behavior. Swarmia, a Software as a Service productivity tool, offers a Working Agreements feature to help teams set and track their ways of working. There is little prior research on how teams utilize such tools and their impact on performance.

\textit{Methods}: The study used a quantitative research design and analyzed data from a sample of approximately five hundred Swarmia client teams. The research aimed to understand the relationship between how teams use Working Agreements and whether the use of Working Agreements influenced software development cycle times. SQL and multiple linear regression were used for the analysis.

\textit{Results}: In the studied sample, $64\%$ of teams configured four or fewer Working Agreements, with the most common agreements being limits for pull request cycle and review times. Limiting the number of open pull requests increased cycle times while prohibiting main branch pushes and limiting review time decreased them. Each team member using Slack notifications reduced the pull request cycle time by $2.5$ hours. Furthermore, using team-level Slack reports reduced the development times. Larger teams miss their targets more often but perform better in terms of pull request cycle time.

\textit{Conclusion}: The study shows that ways of working made with digital tools can positively affect the behavior of software development teams. Multiple relationships were found between the Working Agreement configuration and cycle times. The findings can inform the development of productivity tools and strategies for improving team collaboration. Future research should focus on optimal Working Agreements for different teams, Slack notification optimization, and the impact of agreements that cannot be currently configured in these tools. More metrics and interviews would provide a better view of the teams' situation.
}
% Abstract in Finnish
% ------------------------------------------------------------------
\thesisabstract{finnish}{

\textit{Johdanto}: Tehokas tiimityöskentely on tärkeää ohjelmistokehitysprojektien menestykselle. Tiimityön keskiössä ovat yhteiset työskentelytavat, jotka määritellään tiimin käyttäytymistä ohjaaviksi eksplisiittisiksi ja implisiittisiksi säännöiksi ja normeiksi. Swarmia on SaaS-palvelu tuottavuuden parantamiseen, jonka Working Agreements -ominaisuus auttaa näiden työskentelytapojen sopimiseen ja seurantaan. Aiempaa tutkimusta vastaavien työkalujen käytöstä ja vaikutuksesta tiimien suorituskykyyn on vähäisesti.

\textit{Menetelmät}: Tutkimus toteutettiin määrällisenä tutkimuksena, jossa analysoitiin noin viidensadan Swarmia-asiakastiimin dataa. Tavoitteena oli selvittää Working Agreements -ominaisuuden käytön suhdetta prosessien läpimenoaikoihin. Analyysiin käytettiin SQL:ää ja lineaarista regressiota.

\textit{Tulokset}: Otoksen tiimeistä 64\% määritteli neljä tai vähemmän Working Agreement -sääntöä. Yleisimmät säännöt olivat pull requestien läpimenoajan ja arviointiajan rajoittaminen. Pull requestien määrän rajoittaminen lisäsi läpimenoaikoja, kun taas päähaaraan puskemisen kieltäminen ja arviointiajan rajoittaminen vähensivät niitä. Jokainen Slack-ilmoituksia käyttävä tiimin jäsen vähensi pull requestien läpimenoaikaa 2,5 tunnilla. Lisäksi tiimikohtaiset Slack-raportit vähensivät kehitykseen käytettyä aikaa. Suuremmat tiimit pääsevät tavoitteisiin harvemmin, mutta suoriutuvat paremmin pull requestien läpimenoaikojen osalta.

\textit{Johtopäätökset}: Tutkimus osoittaa, että digitaalisilla työkaluilla määritellyt työskentelytavat voivat vaikuttaa positiivisesti ohjelmistokehitystiimien käyttäytymiseen. Working Agreement -konfiguraatiolla ja läpimenoajoilla havaittiin useita yhteyksiä. Tulokset voivat ohjata tuottavuustyökalujen kehittämistä ja tiimityöskentelyn parantamisponnistuksia. Tulevaisuudessa tutkimuksen tulisi keskittyä erilaisten tiimien optimaalisiin sääntökokoonpanoihin, Slack-ilmoitusten optimointiin sekä niiden sääntöjen vaikutuksen tutkimukseen, joita ei tällä hetkellä voi määritellä näissä työkaluissa. Useampien mittarien ja haastattelujen käyttö antaisi paremman näkymän tiimien tilanteeseen.

}