\chapter{Discussion}



\section{Pohdintaa RQ2}

no direct pushes
    - tiimit tekee muodollisia pr, ei katselmoida
    - “pakko tehdä PR”
    
wip pull requests
    - “tiimi yrittää vähentää monta asiaa on saman aikaisesti katseluputkessa”
    - vähemmän PR listalla, eli vähemmän painetta pistää settejä putkesta läpi → pidempi PRCT
    - koronan ja irtisanomisten vaikutus datasettiin
    
min issue contributors
    - looginen, koska pitää olla useampia contributoreita
    - enemmän väkeä → pidempi CT
    
wip issues
    - sama efekti kuin wip pull requests
    - “wip limitit ei ole kovin hyviä working agreementteja”
    
max issue age
    - PR:t devaajien focus, issue sen sijaan project managerin (laajempi vastuu)
    - kiinnostaa laajemmin PM:ää, katsoo daily digestia jne
    - **minkä tyyppisten issueiden?**
        - “minkä tyyppisissä issueissa saataisiin helpoimmat gainit”
    - voiko olla, että tiimi tarvitsee “potkimista”
    - → jos on buy-in sekä tiimiltä että managementilta, niin saa kaikista merkittävimpiä tuloksia?
    
slack users
    - pitäisikö olla suhteellinen, eli osuus tiimin jäsenistä?
    - isompi kuin swarmia users, koska swarmia users = MAU
    - tää on merkittävä ja mielenkiintoinen
    
github authors
    - onko aktiiviset eli laskutukseen käytettävä?
    - → organisaation koko ei ole vaikuttava tekijä
    - → minkä tahansa kokoisessa tiimissä otettiin ominaisuuksia käyttöön, niin muutosta saatiin aikaan
    
daily digest
    - tiedetään että henk koht notificaatiot on tärkeitä (lähde)
    - tiedetään että WA:t on tehokkaita, kun tiimi on vastuussa niistä ulkopuolelle tai PM:lle (lähde)
    
max pull request age
    - drafteja?
    - tätä katsotaan helpommin läpi sormien devaajien keskuudessa







\section{Limitations}
\todo[inline]{only one category of metrics -> against SPACE principles}
\todo[inline]{harder to get metrics, such as perceived performance, would help with finding a balance}
\todo[inline]{only "fastness of shipping" taken into consideration, what about change failure rate etc (quality)}