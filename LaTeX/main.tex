\documentclass[12pt,a4paper,oneside,pdftex]{report}
\usepackage[utf8]{inputenc}
\usepackage[OT1]{fontenc}
\usepackage[finnish,english]{babel}
%\usepackage{colorprofiles}
%\usepackage[a-2b]{pdfx}

% Natbib allows you to select the format of the bibliography references.
% The first example uses numbered citations: 
\usepackage[square,sort&compress,numbers]{natbib}
% The second example uses author-year citations.
% If you use author-year citations, change the bibliography style (below); 
% acm style does not work with author-year citations.
% Also, you should use \citet (cite in text) when you wish to refer
% to the author directly (\citet{blaablaa} said blaa blaa), and 
% \citep when you wish to refer similarly than with numbered citations
% (It has been said that blaa blaa~\citep{blaablaa}).
% \usepackage[square]{natbib}

% The alltt package provides an all-teletype environment that acts
% like verbatim but you can use LaTeX commands in it. Uncomment if 
% you want to use this environment. 
% \usepackage{alltt}

% The eurosym package provides a euro symbol. Use with \euro{}
\usepackage{eurosym} 

% Verbatim provides a standard teletype environment that renders
% the text exactly as written in the tex file. Useful for code
% snippets (although you can also use the listings package to get
% automatic code formatting). 
\usepackage{verbatim}

% The listing package provides automatic code formatting utilities
% so that you can copy-paste code examples and have them rendered
% nicely. See the package documentation for details.
% \usepackage{listings}

% The fancuvrb package provides fancier verbatim environments 
% (you can, for example, put borders around the verbatim text area
% and so on). See package for details.
% \usepackage{fancyvrb}

% Supertabular provides a tabular environment that can span multiple 
% pages. 
%\usepackage{supertabular}
% Longtable provides a tabular environment that can span multiple 
% pages. This is used in the example acronyms file. 
\usepackage{longtable}

% The fancyhdr package allows you to set your the page headers 
% manually, and allows you to add separator lines and so on. 
% Check the package documentation. 
% \usepackage{fancyhdr}

% Subfigure package allows you to use subfigures (i.e. many subfigures
% within one figure environment). These can have different labels and
% they are numbered automatically. Check the package documentation. 
\usepackage{subfigure}

% The titlesec package can be used to alter the look of the titles 
% of sections, chapters, and so on. This example uses the ``medium'' 
% package option which sets the titles to a medium size, making them
% a bit smaller than what is the default. You can fine-tune the 
% title fonts and sizes by using the package options. See the package
% documentation.
\usepackage[medium]{titlesec}
\usepackage{listings}

% The TikZ package allows you to create professional technical figures.
% The learning curve is quite steep, but it is definitely worth it if 
% you wish to have really good-looking technical figures. 
\usepackage{tikz}
% You also need to specify which TikZ libraries you use
\usetikzlibrary{positioning}
\usetikzlibrary{calc}
\usetikzlibrary{arrows}
\usetikzlibrary{decorations.pathmorphing,decorations.markings}
\usetikzlibrary{shapes}
\usetikzlibrary{patterns}


% The aalto-thesis package provides typesetting instructions for the
% standard master's thesis parts (abstracts, front page, and so on)
% Load this package second-to-last, just before the hyperref package.
% Options that you can use: 
%   mydraft - renders the thesis in draft mode. 
%             Do not use for the final version. 
%   doublenumbering - [optional] number the first pages of the thesis
%                     with roman numerals (i, ii, iii, ...); and start
%                     arabic numbering (1, 2, 3, ...) only on the 
%                     first page of the first chapter
\usepackage{aalto-thesis}
%\usepackage[mydraft,doublenumbering]{aalto-thesis}



% Hyperref
% ------------------------------------------------------------------
% Hyperref creates links from URLs, for references, and creates a
% TOC in the PDF file.
% This package must be the last one you include, because it has
% compatibility issues with many other packages and it fixes
% those issues when it is loaded.   
%\RequirePackage[pdftex]{hyperref}
\RequirePackage[pdfa]{hyperref}
% Setup hyperref so that links are clickable but do not look 
% different
\hypersetup{colorlinks=false,raiselinks=false,breaklinks=true}
\hypersetup{pdfborder={0 0 0}}
\hypersetup{bookmarksnumbered=true}
% The following line suggests the PDF reader that it should show the 
% first level of bookmarks opened in the hierarchical bookmark view. 
\hypersetup{bookmarksopen=true,bookmarksopenlevel=1}
% Hyperref can also set up the PDF metadata fields. These are
% set a bit later on, after the thesis setup.   


% Thesis setup
% ==================================================================
% If you do not find the command for a text that is shown in the cover page or
% in the abstract texts, check the aalto-thesis.sty file and locate the text
% from there. 
% All the texts are configured in language-specific blocks (lots of commands
% that look like this: \renewcommand{\ATCITY}{Espoo}.
% You can just fix the texts there. Just remember to check all the language
% variants you use (they are all there in the same place). 
% ------------------------------------------------------------------
\newcommand{\TITLE}{}
\newcommand{\DATE}{May 1, 2022}

\newcommand{\FTITLE}{}
\newcommand{\FDATE}{Toukokuu 1, 2022}

% Supervisors and instructors
% ------------------------------------------------------------------
% Usually thesis have one supervisor and one advisor. Sometimes you
% may have two advisors and, in double degree
% programs, you may have two supervisors. 
% If you have two supervisors, write both names here, separate them with a 
% double-backslash (see below for an example)
% Also remember to add the package option ``twosupervisors'' or
% ``twoinstructors'' to the aalto-thesis package (aalto-thesis.sty
% file line 72), so that the titles are in plural.

% Example of one supervisor:
\newcommand{\SUPERVISOR}{Professor X Y}
\newcommand{\FSUPERVISOR}{Professori A B}

% If you have only one instructor, just write one name here
\newcommand{\INSTRUCTOR}{M.Sc. (Tech.) A B}
\newcommand{\FINSTRUCTOR}{Diplomi-insinööri A B}

% Other stuff
% ------------------------------------------------------------------
\newcommand{\PROFESSORSHIP}{Security and Cloud Computing}
\newcommand{\FPROFESSORSHIP}{Tietoturva ja pilvilaskenta}
% Professorship code is the same in all languages
\newcommand{\PROFCODE}{SCI3084}
\newcommand{\KEYWORDS}{Kata Containers, Kubernetes, Far Edge Cloud, Mobile Edge Computing, Container, Security, I/O performance}
\newcommand{\FKEYWORDS}{Kata Containers, Kubernetes, Far Edge Cloud, mobiiliverkko, ohjelmistokontti, tietoturva, suorituskyky}
\newcommand{\LANGUAGE}{English}
\newcommand{\FLANGUAGE}{Englanti}

% Author is the same for all languages
\newcommand{\AUTHOR}{Petrus Holm}

% Currently the English versions are used for the PDF file metadata
% Set the PDF title
\hypersetup{pdftitle={\TITLE}}
% Set the PDF author
\hypersetup{pdfauthor={\AUTHOR}}
% Set the PDF keywords
\hypersetup{pdfkeywords={\KEYWORDS}}
% Set the PDF subject
\hypersetup{pdfsubject={Master's Thesis}}


% Layout settings
% ------------------------------------------------------------------

% When you write in English, you should use the standard LaTeX 
% paragraph formatting: paragraphs are indented, and there is no 
% space between paragraphs.
% When writing in Finnish, we often use no indentation in the
% beginning of the paragraph, and there is some space between the 
% paragraphs.

% Use this to control how much space there is between each line of text.
% 1 is normal (no extra space), 1.3 is about one-half more space, and
% 1.6 is about double line spacing.  
% \linespread{1} % This is the default
% \linespread{1.3}

% Bibliography style
% acm style gives you a basic reference style. It works only with numbered
% references.
\bibliographystyle{acm}
% Plainnat is a plain style that works with both numbered and name citations.
% \bibliographystyle{plainnat}


% Extra hyphenation settings
% ------------------------------------------------------------------
% You can list here all the files that are not hyphenated correctly.
% You can provide many \hyphenation commands and/or separate each word
% with a space inside a single command. Put hyphens in the places where
% a word can be hyphenated.
% Note that (by default) LaTeX will not hyphenate words that already
% have a hyphen in them (for example, if you write ``structure-modification 
% operation'', the word structure-modification will never be hyphenated).
% You need a special package to hyphenate those words.
\hyphenation{di-gi-taa-li-sta yksi-suun-tai-sta}



% The preamble ends here, and the document begins. 
% Place all formatting commands and such before this line.
% ------------------------------------------------------------------
\begin{document}
% This command adds a PDF bookmark to the cover page. You may leave
% it out if you don't like it...
\pdfbookmark[0]{Cover page}{bookmark.0.cover}
% This command is defined in aalto-thesis.sty. It controls the page 
% numbering based on whether the double-numbering option is specified
\startcoverpage

% Cover page
% These control in which language the cover-page information is shown
\pagenumbering{roman}
\coverpage{english}



% Abstracts
% ------------------------------------------------------------------
% Include an abstract in the language that the thesis is written in,
% and if your native language is Finnish or Swedish, one in that language.

%\input{0abstract.tex}


% Acknowledgements
% ------------------------------------------------------------------
% Select the language you use in your acknowledgements
\selectlanguage{english}

% Uncomment this line if you wish acknowledgements to appear in the 
% table of contents
%\addcontentsline{toc}{chapter}{Acknowledgements}

% The star means that the chapter isn't numbered and does not 
% show up in the TOC
\chapter*{Acknowledgements}



\vskip 10mm

\noindent 
\FDATE
\vskip 5mm
\noindent\AUTHOR

% Acronyms
% ------------------------------------------------------------------
% Use \cleardoublepage so that IF two-sided printing is used 
% (which is not often for masters theses), then the pages will still
% start correctly on the right-hand side.
\cleardoublepage
% Example acronyms are placed in a separate file, acronyms.tex
%\addcontentsline{toc}{chapter}{Abbreviations and Acronyms}
\chapter*{Abbreviations and Acronyms}

\noindent
\begin{longtable}{@{}p{0.25\textwidth}p{0.7\textwidth}@{}}
CD & Continuous Delivery \\
CI & Continuous Integration \\
DORA & DevOps Research and Assessment \\
IEEE & Institute of Electrical and Electronics Engineers \\
IPCT & In Progress Cycle Time \\
IRCT & In Review Cycle Time \\
PR & Pull Request \\
PRCT & Pull Request Cycle Time \\
RMCT & Ready to Merge Cycle Time \\
SDLC & Software Development Life Cycle \\
WA & Working Agreement \\
WIP & Work In Progress \\
\end{longtable}

% Table of contents
% ------------------------------------------------------------------
\cleardoublepage
% This command adds a PDF bookmark that links to the contents.
% You can use \addcontentsline{} as well, but that also adds contents
% entry to the table of contents, which is kind of redundant.
% The text ``Contents'' is shown in the PDF bookmark. 
\pdfbookmark[0]{Contents}{bookmark.0.contents}
\tableofcontents

% List of tables
% ------------------------------------------------------------------
% You only need a list of tables for your thesis if you have very 
% many tables. If you do, uncomment the following two lines.
% \cleardoublepage
% \listoftables

% Table of figures
% ------------------------------------------------------------------
% You only need a list of figures for your thesis if you have very 
% many figures. If you do, uncomment the following two lines.
% \cleardoublepage
% \listoffigures

% The following label is used for counting the prelude pages
\label{pages-prelude}
\cleardoublepage

%%%%%%%%%%%%%%%%% The main content starts here %%%%%%%%%%%%%%%%%%%%%
% ------------------------------------------------------------------
% This command is defined in aalto-thesis.sty. It controls the page 
% numbering based on whether the double-numbering option is specified
\startfirstchapter
\pagenumbering{arabic}

% Add headings to pages (the chapter title is shown)
\pagestyle{headings}


% The contents of the thesis are separated to their own files.
% Edit the content in these files, rename them as necessary.
% ------------------------------------------------------------------


% Load the bibliographic references
% ------------------------------------------------------------------
% You can use several .bib files:
% \bibliography{thesis_sources,ietf_sources}
\bibliography{sources}

\appendix
%\input{9appendix.tex}

\end{document}