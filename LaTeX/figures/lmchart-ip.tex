% another way of presenting linear model results, maybe better than the table. How to show p value here?å
%  metrics 9-11 are not boolean (amount of people), so they should be presented in other way.
\begin{filecontents}{linearmodel-ip.data}
type    amount      p
1    1.55262826067878    ***
2    -0.234697922043722   $\nan$	
3    -1.00973405990083    ***
4    0.682170335849312    $\nan$	
5    0.698565767442738   $\nan$
6    1.07083700224552     ***
7    -1.50650881760295    ***
8    -0.252902503872395    $\nan$	
9    -0.0772665164223861   $\nan$	
10    -0.595349555897268    **
11    1                    $\nan$
\end{filecontents}

\begin{figure}[ht]
\centering
\begin{tikzpicture}
\begin{axis}[
	x tick label style={
		 rotate=45, anchor=east},
	ylabel=Effect on IPCT,
        xticklabels={wip\_pull\_requests,
                    max\_pull\_request\_age,
                    no\_direct\_pushes\_to\_main\_branch,
                    min\_issue\_contributors,
                    wip\_issues,
                    max\_issue\_age,
                    max\_pull\_request\_review\_time,
                    pull\_request\_linking,
                    slack\_users,
                    daily\_digest
                    },
	enlargelimits=0.05,
	legend style={at={(0.5,-0.1)},
	anchor=north,legend columns=-1},
        nodes near coords,
        every node near coord/.append style={anchor=west},
        point meta=explicit symbolic,
        enlarge y limits=0.2,
	ybar interval=0.8
        ]
\addplot[fill=cyan]table[x=type, y=amount, meta=p]{linearmodel-ip.data};
\end{axis}
\end{tikzpicture}
\caption{Linear regression results for IPCT}
\label{fig:lmInProgressChart}
\end{figure}
