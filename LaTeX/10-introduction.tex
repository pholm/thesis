\chapter{Introduction}

Technology companies are looking for ways to get the most out of their software development teams in the highly competitive market. Previously, the focus has been on the performance of individual developers with the help of artificial metrics, such as lines of code. 

A new wave of productivity tools has emerged to answer this need. Many of these products claim to be based on peer-reviewed research. The tools are openly opinionated and aim to influence their client companies to use specific popular development methodologies, such as Continuous Integration. Companies building solutions in the field include LinearB, Haystack, Jellyfish, and Swarmia.

Swarmia is a Finnish startup company that has developed its namesake developer productivity tool since 2019. The product vision is based on the idea that development teams use the tool for self-improvement instead of providing another dashboard for upper management. The application integrates with version control and project planning systems to provide insights to the teams. 

Working Agreements is a key feature of Swarmia. In Working Agreements, teams configure rules into Swarmia. Then, the application starts tracking them and informing the team members of their performance. For instance, a team can set limits to open issues or pull requests. The pre-defined set of agreement templates is based on standard software development practices. 

\section{Problem statement}

As a null hypothesis, we assume that Working Agreements do not influence team productivity. This hypothesis is then challenged using two research questions: 

\begin{enumerate}
    \item[{\bf RQ1}] How do teams use Working Agreements?
    \item[{\bf RQ2}] How do Working Agreements influence team productivity?
\end{enumerate}

For RQ1, we examine the usage data of Swarmia teams. The aim is to determine how popular the Working Agreements feature is and how the teams have configured it. 

We look into RQ2 from two aspects: Do Working Agreements overall influence team productivity and if they do, are some configurations of working agreements more influential than others? 

\section{Scope and Methodology}



\section{Results}

\todo{short summary of results based on 40-results}
