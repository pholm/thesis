
\chapter{Results}

\section{How teams use Working Agreements}


To get an overview on how teams are using Working Agreements, we started with the proportional popularity of WAs. As can be seen in Figure \ref{fig:waPopularity}, the agreements \textit{max\_pull\_request\_review\_time} and \textit{max\_pull\_request\_age} are used the most, with over three hundred teams. Another atypical WA is \textit{min\_issue\_contributors}, with approximately 150 teams. The rest of the Working Agreements are used by approximately two hundred teams each.

\begin{filecontents}{usage.data}
type amount name
1 219 wip pull
2 315 max pr
3 229 no dir
4 148 min issue
5 204 wip iss
6 225 max issue
7 341 max pr 
8 206 pr linking
9 111 empty
\end{filecontents}

\begin{figure}[h]
\centering
\begin{tikzpicture}
\begin{axis}[
	x tick label style={
		 rotate=45, anchor=east},
	ylabel=Number of teams using,
        xticklabels={\texttt{\detokenize{wip_pull_requests}},
                    \texttt{\detokenize{max_pull_request_age}},
                    \texttt{\detokenize{no_direct_pushes_to_main_branch}},
                    \texttt{\detokenize{min_issue_contributors}},
                    \texttt{\detokenize{wip_issues}},
                    \texttt{\detokenize{max_issue_age}},
                    \texttt{\detokenize{max_pull_request_review_time}},
                    \texttt{\detokenize{pull_request_linking}}},
	enlargelimits=0.05,
	legend style={at={(0.5,-0.1)},
	anchor=north,legend columns=-1},
	ybar interval=0.8,
        ymin=0
]
\addplot[fill=cyan] table[x=type, y=amount] {usage.data};
\end{axis}
\end{tikzpicture}
\caption{Working Agreement usage}
\label{fig:waPopularity}
\end{figure}

A distribution of how many WAs each team has in use was calculated. The results are plotted in Figure \ref{fig:waDistribution}. Even though there exists eight Working Agreement templates, the number of WAs in use can exceed this: issue related agreements can be configured multiple times for a single team. For example, \textit{max\_issue\_age} can be configured distinctly for epics, stories, tasks and bugs. Still, only a small share of teams have configured more than eight WAs.

\begin{filecontents}{wa.data}
was teams
1 23.55
2 15.08
3 13.64
4 11.36
5 12.60
6 6.82
7 7.02
8 5.58
9 1.65
10 1.24
11 0.62
12 0.62
13 0.21
14 1
\end{filecontents}

\begin{figure}
\centering
\begin{tikzpicture}
\begin{axis}[
	x tick label style={
		/pgf/number format/1000 sep=},
	ylabel=Percentage of teams,
        xlabel=Amount of WAs,
	enlargelimits=0.05,
	legend style={at={(0.5,-0.1)},
	anchor=north,legend columns=-1},
        yticklabel={\pgfmathparse{\tick}\pgfmathprintnumber{\pgfmathresult}\%},
	ybar interval=0.8,
]
\addplot[fill=cyan] table[x=was, y=teams] {wa.data};
\end{axis}
\end{tikzpicture}
\caption{Teams' WA usage distribution}
\label{fig:waDistribution}
\end{figure}

Finally, the usage of issue related Working Agreements was looked into. Figure \ref{fig:issue_was} shows that for all three templates, it is quite rare for a team to configure more than one Working Agreement. The distribution of issue types in these three Working Agreements is shown in Table \ref{tab:issueConfig}. Story is by far the most popular issue type with 359 configured Working Agreements. On the other hand, bug is the least used with only 36 WAs, a huge difference to story.

\begin{table}[h]
\begin{center}
\begin{tabular}{|c|c|c|c|c|} 
\hline

WA & Story & Bug & Epic & Task \\ [0.5ex] 
\hline\hline

max\_issue\_age & 155 & 29 & 12 & 33 \\
min\_issue\_contributors & 52 & 1 & 75 & 10 \\
wip\_issues & 152 & 6 & 29 & 32 \\
\textbf{Grand total} & \textbf{359} & \textbf{36} & \textbf{116} & \textbf{75} \\
\hline
\end{tabular}
\caption{Issue type distribution}
\label{tab:issueConfig}
\end{center}
\end{table}

\todo[inline]{configs of working agreements?}
\todo[inline]{if no interesting config connection, remove issue was}
\begin{filecontents}{wip_issues.data}
was teams
1   120
2   30
3   6
4   5
5   5
\end{filecontents}

\begin{filecontents}{issue_ct.data}
was teams
1   150
2   21
3   3
4   2
5   5
\end{filecontents}

\begin{filecontents}{min_contribs.data}
was teams
1   115
2   12
3   1
4   1
5   5
\end{filecontents}

\begin{figure}[h]
\begin{subfigure}{.25\textwidth}
    \centering
    \begin{tikzpicture}
    \begin{axis}[
    	x tick label style={
    		/pgf/number format/1000 sep=},
    	ylabel=Number of teams,
            xlabel=Amount of options,
    	enlargelimits=0.05,
    	legend style={at={(0.5,-0.1)},
    	anchor=north,legend columns=-1},
    	ybar interval=0.8,
            width=\linewidth,
    ]
    \addplot[fill=cyan] table[x=was, y=teams] {wip_issues.data};
    \end{axis}
    \end{tikzpicture}
    \caption{wip\_issues}
\end{subfigure}
\hfill
\begin{subfigure}{.25\textwidth}
    \centering
    \begin{tikzpicture}
    \begin{axis}[
    	x tick label style={
    		/pgf/number format/1000 sep=},
    	ylabel=Number of teams,
            xlabel=Amount of options,
    	enlargelimits=0.05,
    	legend style={at={(0.5,-0.1)},
    	anchor=north,legend columns=-1},
    	ybar interval=0.8,
            width=\linewidth,
    ]
    \addplot[fill=cyan] table[x=was, y=teams] {issue_ct.data};
    \end{axis}
    \end{tikzpicture}
    \caption{max\_issue\_age}
\end{subfigure}
\hfill
\begin{subfigure}{.25\textwidth}
    \centering
    \begin{tikzpicture}
    \begin{axis}[
    	x tick label style={
    		/pgf/number format/1000 sep=},
    	ylabel=Number of teams,
            xlabel=Amount of options,
    	enlargelimits=0.05,
    	legend style={at={(0.5,-0.1)},
    	anchor=north,legend columns=-1},
    	ybar interval=0.8,
            width=\linewidth,
    ]
    \addplot[fill=cyan] table[x=was, y=teams] {min_contribs.data};
    \end{axis}
    \end{tikzpicture}
    \caption{min\_issue\_contr}
\end{subfigure}
\caption{Issue related WA usage}
\label{fig:issue_was}
\end{figure}






\section{Working Agreements and team productivity}

The main method used to answer RQ2 is the multiple linear regression model. The aim was to find out how Working Agreement configurations and other independent factors affect PRCT. The results are presented in Table \ref{tab:lmResults}. The values in "Iteration" columns represent weight of the variable in days: a negative values therefore imply decrease in PRCT, while positive values imply increase. Additionally, the same information is visualised in Figure \ref{fig:lmResultsChart}. 

As can be seen in the Figure, \textit{max\_issue\_age} and \textit{no\_direct\_pushes\_to\_main\_branch} have the most significant decreasing effect on PRCT. On the other hand, \textit{min\issue\_contributors}, \textit{wip\_issues} and \textit{wip\_pull\_requests} have the highest increasing impact. All of these results have statistical relevance, with $ p < 0.10 $.

\textit{slack\_users} has weight -0.196 with $ p < 0.01 $. It's relevant to point out that while WA related variables have Boolean values, user related metrics don't: they are measured in absolute team members. Therefore, while the \textit{slack\_users} weight is relatively small, it can have a larger impact on the PRCT than a single WA configuration.

% another way of presenting linear model results, maybe better than the table. How to show p value here?å
%  metrics 9-11 are not boolean (amount of people), so they should be presented in other way.
\begin{filecontents}{linearmodel.data}
type    amount      p
1    1.28059346644558    ***	
2    -0.36837516451397    \nan	
3    -1.2810216722553    ***	
4    0.599425898551173    \nan	
5    0.544105176082035    \nan	
6    -0.385152907351116    \nan	
7    -0.812495250694635    *	
8    0.450854667252589    \nan	
9    -0.101656544719984    **	
10    0.348822777606775    \nan	
11    1                   \nan
\end{filecontents}

\begin{figure}[h]
\centering
\begin{tikzpicture}
\begin{axis}[
	x tick label style={
		 rotate=45, anchor=east},
	ylabel=Effect on PRCT,
        xticklabels={\texttt{\detokenize{wip_pull_requests}},
                    \texttt{\detokenize{max_pull_request_age}},
                    \texttt{\detokenize{no_direct_pushes_to_main_branch}},
                    \texttt{\detokenize{min_issue_contributors}},
                    \texttt{\detokenize{wip_issues}},
                    \texttt{\detokenize{max_issue_age}},
                    \texttt{\detokenize{max_pull_request_review_time}},
                    \texttt{\detokenize{pull_request_linking}},
                     \texttt{\detokenize{slack_users}},
                     \texttt{\detokenize{daily_digest}}},
	enlargelimits=0.05,
	legend style={at={(0.5,-0.1)},
	anchor=north,legend columns=-1},
        nodes near coords,
        every node near coord/.append style={anchor=west},
        point meta=explicit symbolic,
        enlarge y limits=0.2,
	ybar interval=0.8,
        width=\textwidth
]
\addplot[fill=cyan] table[x=type, y=amount, meta=p] {linearmodel.data};
\end{axis}
\end{tikzpicture}
\caption{Linear regression results, one asterisk \texttt{*} denotes  $p < 0.10$, two asterisks denotes $p < 0.05$, and three asterisks denote $p < 0.01$.}
\label{fig:lmResultsChart}
\end{figure}


\renewcommand{\arraystretch}{1.2}
\begin{table}
\begin{center}
\begin{tabular}{|p{6cm}|p{3cm}|p{3cm}|} 
\hline
variable & weight & p-value \\ [0.5ex]
\hline\hline

wip\_pull\_requests & $1.28^{***}$ & $0.000402$ \\
max\_pull\_request\_age & $-0.368$ & $0.318$ \\
no\_direct\_pushes\_to\_main\_branch & $-1.28^{***}$ & $0.000166$ \\
min\_issue\_contributors & $0.599$ & $0.128$ \\
wip\_issues & $0.544$ & $0.17$ \\
max\_issue\_age & $-0.385$ & $0.299$ \\
max\_pull\_request\_review\_time & $-0.812^{**}$ & $0.0238$ \\
pull\_request\_linking & $0.451$ & $0.197$ \\
slack\_users & $-0.102^{***}$ & $0.00208$ \\
daily\_digest & $0.349$ & $0.224$ \\

\hline\hline
INTERCEPT & $1.38^{***}$ & $4.84*10^{-6}$ \\ 

\hline
\end{tabular}
\caption{Linear regression results for PRCT, one asterisk \texttt{*} denotes  $p < 0.10$, two asterisks denotes $p < 0.05$, and three asterisks denote $p < 0.01$.}
\label{tab:lmResults}
\end{center}
\end{table}

In progress cycle time results can be seen in Figure \ref{fig:lmInProgressChart}. The direction of effect is the same for most variables as in PRCT model. 

\todo{deeper analysis}

% another way of presenting linear model results, maybe better than the table. How to show p value here?å
%  metrics 9-11 are not boolean (amount of people), so they should be presented in other way.
\begin{filecontents}{linearmodel_ip.data}
type    amount      p
1    1.55262826067878    ***
2    -0.234697922043722    \nan
3    -1.00973405990083    **
4    0.682170335849312    \nan
5    0.698565767442738    \nan
6    1.07083700224552    **
7    -1.50650881760295    ***
8    -0.252902503872395    \nan
9    -0.0772665164223861    \nan
10    -0.595349555897268    *
11    1                    \nan
\end{filecontents}

\begin{figure}[h]
\centering
\begin{tikzpicture}
\begin{axis}[
	x tick label style={
		 rotate=45, anchor=east},
	ylabel=Effect on IPCT,
        xticklabels={\texttt{\detokenize{wip_pull_requests}},
                    \texttt{\detokenize{max_pull_request_age}},
                    \texttt{\detokenize{no_direct_pushes_to_main_branch}},
                    \texttt{\detokenize{min_issue_contributors}},
                    \texttt{\detokenize{wip_issues}},
                    \texttt{\detokenize{max_issue_age}},
                    \texttt{\detokenize{max_pull_request_review_time}},
                    \texttt{\detokenize{pull_request_linking}},
                     \texttt{\detokenize{slack_users}},
                     \texttt{\detokenize{daily_digest}}},
	enlargelimits=0.05,
	legend style={at={(0.5,-0.1)},
	anchor=north,legend columns=-1},
        nodes near coords,
        every node near coord/.append style={anchor=west},
        point meta=explicit symbolic,
        enlarge y limits=0.2,
	ybar interval=0.8
]
\addplot[fill=cyan] table[x=type, y=amount, meta=p] {linearmodel_ip.data};
\end{axis}
\end{tikzpicture}
\caption{Linear regression results for IPCT}
\label{fig:lmInProgressChart}
\end{figure}
