
\chapter{Results}

\section{How teams use Working Agreements}


To get an overview of how teams are using Working Agreements, we started with the proportional popularity of WAs. As can be seen in Figure \ref{fig:waPopularity}, the agreements max\_pull\_request\_review\_time and max\_pull\_request\_age are used the most, with over three hundred teams. Another atypical WA is min\_issue\_contributors, with approximately 150 teams. The rest of the Working Agreements are used by approximately two hundred teams each.

\input{LaTeX/figures/wa_usage.tex}

The distribution of how many WAs each team has in use was calculated. The results are plotted in Figure \ref{fig:waDistribution}. Even though eight Working Agreement templates exist, the number of WAs in use can exceed this: issue-related agreements can be configured multiple times for a single team. For example, max\_issue\_age can be configured distinctly for epics, stories, tasks, and bugs. Still, only a tiny share of teams have configured more than eight WAs.

\begin{filecontents}{wa.data}
was teams
1 114
2 73
3 66
4 55
5 61
6 33
7 34
8 27
9 8
10 6
11 3
12 3
13 1
14 1
\end{filecontents}

\begin{figure}[h]
\centering
\begin{tikzpicture}
\begin{axis}[
	x tick label style={
		/pgf/number format/1000 sep=},
	ylabel=Number of teams,
        xlabel=Amount of WAs,
	enlargelimits=0.05,
	legend style={at={(0.5,-0.1)},
	anchor=north,legend columns=-1},
	ybar interval=0.8,
]
\addplot[fill=cyan] table[x=was, y=teams] {wa.data};
\end{axis}
\end{tikzpicture}
\caption{Teams' WA usage distribution}
\label{fig:waDistribution}
\end{figure}

Finally, the usage of issue-related Working Agreements was looked into. Figure \ref{fig:issue_was} shows that for all three templates, it is pretty rare for a team to configure more than one Working Agreement. The distribution of issue types in these three Working Agreements is shown in Table \ref{tab:issueConfig}. The story is the most popular issue type, with 359 configured Working Agreements. On the other hand, bug is the least used, with only 36 WAs, a stark contrast to story.

\begin{table}
\begin{center}
\begin{tabular}{|c|c|c|c|c|} 
\hline

WA & Story & Bug & Epic & Task \\ [0.5ex] 
\hline\hline

max\_issue\_age & 155 & 29 & 12 & 33 \\
min\_issue\_contributors & 52 & 1 & 75 & 10 \\
wip\_issues & 152 & 6 & 29 & 32 \\
\textbf{Grand total} & \textbf{359} & \textbf{36} & \textbf{116} & \textbf{75} \\
\hline
\end{tabular}
\caption{Issue type distribution}
\label{tab:issueConfig}
\end{center}
\end{table}

\begin{filecontents}{wip_issues.data}
was teams
1   120
2   30
3   6
4   5
5   5
\end{filecontents}

\begin{filecontents}{issue_ct.data}
was teams
1   150
2   21
3   3
4   2
5   5
\end{filecontents}

\begin{filecontents}{min_contribs.data}
was teams
1   115
2   12
3   1
4   1
5   5
\end{filecontents}

\begin{figure}[h]
\begin{subfigure}{.25\textwidth}
    \centering
    \begin{tikzpicture}
    \begin{axis}[
    	x tick label style={
    		/pgf/number format/1000 sep=},
    	ylabel=Number of teams,
            xlabel=Amount of options,
    	enlargelimits=0.05,
    	legend style={at={(0.5,-0.1)},
    	anchor=north,legend columns=-1},
    	ybar interval=0.8,
            width=\linewidth,
    ]
    \addplot[fill=cyan] table[x=was, y=teams] {wip_issues.data};
    \end{axis}
    \end{tikzpicture}
    \caption{wip\_issues}
\end{subfigure}
\hfill
\begin{subfigure}{.25\textwidth}
    \centering
    \begin{tikzpicture}
    \begin{axis}[
    	x tick label style={
    		/pgf/number format/1000 sep=},
    	ylabel=Number of teams,
            xlabel=Amount of options,
    	enlargelimits=0.05,
    	legend style={at={(0.5,-0.1)},
    	anchor=north,legend columns=-1},
    	ybar interval=0.8,
            width=\linewidth,
    ]
    \addplot[fill=cyan] table[x=was, y=teams] {issue_ct.data};
    \end{axis}
    \end{tikzpicture}
    \caption{max\_issue\_age}
\end{subfigure}
\hfill
\begin{subfigure}{.25\textwidth}
    \centering
    \begin{tikzpicture}
    \begin{axis}[
    	x tick label style={
    		/pgf/number format/1000 sep=},
    	ylabel=Number of teams,
            xlabel=Amount of options,
    	enlargelimits=0.05,
    	legend style={at={(0.5,-0.1)},
    	anchor=north,legend columns=-1},
    	ybar interval=0.8,
            width=\linewidth,
    ]
    \addplot[fill=cyan] table[x=was, y=teams] {min_contribs.data};
    \end{axis}
    \end{tikzpicture}
    \caption{min\_issue\_contr}
\end{subfigure}
\caption{Issue related WA usage}
\label{fig:issue_was}
\end{figure}






\section{Working Agreements and team productivity}

The primary method used to answer RQ2 is the multiple linear regression model. The aim was to determine how Working Agreement configurations and other independent factors affect PRCT. The results are presented in Table \ref{tab:lmResults}. The values in the "Iteration" columns represent the weight of the variable in days: negative values, therefore, imply a decrease in PRCT, while positive values imply an increase. Additionally, the same information is visualized in Figure \ref{fig:lmResultsChart}. 

max\_issue\_age and no\_direct\_pushes\_to\_main\_branch have the most significant decreasing effect on PRCT. On the other hand, min\_issue\_contributors, wip\_issues and wip\_pull\_requests have the highest increasing impact. These results have statistical relevance, with $ p < 0.10 $.

slack\_users has weight -0.196 with $ p < 0.01 $. It is relevant to point out that while WA-related variables have Boolean values, user-related metrics don't: they are measured in absolute team members. Therefore, while the slack\_users weight is relatively small, it can have a larger impact on the PRCT than a single WA configuration.

\begin{filecontents}{linearmodel.data}
type    amount      p
1    1.28059346644558    ***	
2    -0.36837516451397    \nan	
3    -1.2810216722553    ***	
4    0.599425898551173    \nan	
5    0.544105176082035    \nan	
6    -0.385152907351116    \nan	
7    -0.812495250694635    **	
8    0.450854667252589    \nan	
9    -0.101656544719984    ***	
10    0.348822777606775    \nan	
11    1                   \nan
\end{filecontents}

\begin{figure}
\centering
\begin{tikzpicture}
\begin{axis}[
	x tick label style={
		 rotate=45, anchor=east},
	ylabel=Effect on PRCT,
        xticklabels={\texttt{\detokenize{wip_pull_requests}},
                    \texttt{\detokenize{max_pull_request_age}},
                    \texttt{\detokenize{no_direct_pushes_to_main_branch}},
                    \texttt{\detokenize{min_issue_contributors}},
                    \texttt{\detokenize{wip_issues}},
                    \texttt{\detokenize{max_issue_age}},
                    \texttt{\detokenize{max_pull_request_review_time}},
                    \texttt{\detokenize{pull_request_linking}},
                     \texttt{\detokenize{slack_users}},
                     \texttt{\detokenize{daily_digest}}},
	enlargelimits=0.05,
	legend style={at={(0.5,-0.1)},
	anchor=north,legend columns=-1},
        nodes near coords,
        every node near coord/.append style={anchor=west},
        point meta=explicit symbolic,
        enlarge y limits=0.2,
	ybar interval=0.8
]
\addplot[fill=cyan] table[x=type, y=amount, meta=p] {linearmodel.data};
\end{axis}
\end{tikzpicture}
\caption{Linear regression results for PRCT, one asterisk \texttt{*} denotes  $p < 0.10$, two asterisks denotes $p < 0.05$, and three asterisks denote $p < 0.01$.}
\label{fig:lmResultsChart}
\end{figure}


\renewcommand{\arraystretch}{1.2}
\begin{table}[h]
\begin{center}
\begin{tabular}{|p{6cm}|p{4cm}|p{4cm}|} 
\hline
variable & weight & p-value \\ [0.5ex]
\hline\hline

wip\_pull\_requests & $1.28^{***}$ & $0.000402$ \\
max\_pull\_request\_age & $-0.368$ & $0.318$ \\
no\_direct\_pushes\_to\_main\_branch & $-1.28^{***}$ & $0.000166$ \\
min\_issue\_contributors & $0.599$ & $0.128$ \\
wip\_issues & $0.544$ & $0.17$ \\
max\_issue\_age & $-0.385$ & $0.299$ \\
max\_pull\_request\_review\_time & $-0.812^{*}$ & $0.0238$ \\
pull\_request\_linking & $0.451$ & $0.197$ \\
slack\_users & $-0.102^{**}$ & $0.00208$ \\
daily\_digest & $0.349$ & $0.224$ \\

\hline
\end{tabular}
\caption{Linear regression results, one asterisk \texttt{*} denotes  $p < 0.10$, two asterisks denotes $p < 0.05$, and three asterisks denote $p < 0.01$.}
\label{tab:lmResults}
\end{center}
\end{table}


% wip\_pull\_requests & $1.28^{***} \newline (p=0.000402)$ \\
% max\_pull\_request\_age & $-0.368^{} \newline (p=0.318)$ \\
% no\_direct\_pushes\_to\_main\_branch & $-1.28^{***} \newline (p=0.000166)$ \\
% min\_issue\_contributors & $0.599^{} \newline (p=0.128)$ \\
% wip\_issues & $0.544^{} \newline (p=0.17)$ \\
% max\_issue\_age & $-0.385^{} \newline (p=0.299)$ \\
% max\_pull\_request\_review\_time & $-0.812^{*} \newline (p=0.0238)$ \\
% pull\_request\_linking & $0.451^{} \newline (p=0.197)$ \\
% slack\_users & $-0.102^{**} \newline (p=0.00208)$ \\
% daily\_digest & $0.349^{} \newline (p=0.224)$ \\


In progress, cycle time results can be seen in Figure \ref{fig:lmInProgressChart}. The direction of effect is the same for most variables as in the PRCT model. 

\todo{deeper analysis}

\input{LaTeX/figures/lmchart_ip.tex}