\chapter{Conclusion}

In this thesis we explored the use of Working Agreements and whether it influenced software development cycle times. The study was conducted using data from the Swarmia platform, which is a SaaS product teams use to improve their performance. Swarmia feature called Working Agreements and its relation to team performance was analyzed. The analyses were quantitative, where the way teams configure Working Agreements and the configuration's relationship with the team's software development cycle times were researched. The methods used included SQL aggregates and multiple linear regression models. 

To summarize, the research questions and their answers are as follows. 

For RQ1, ``How do teams use Working Agreements?'', we observed that a majority of teams configured four or fewer Working Agreements. The most popular WAs are limits for PR cycle time and PR review time. Regarding issue types, teams tend to gravitate towards Working Agreements that deal with stories. 

For RQ2, ``What is the effect of Working Agreements on software development cycle times?'' we observed that PRCT and IPCT are shown to increase when teams limit the number of open pull requests. On the other hand, discouraging direct pushes to the main branch and limiting pull request review times decreased these metrics. Regarding team characteristics, more team members using Slack notifications reduced the PRCT, and using Daily Digest, a team-level daily report, reduced IPCT. Furthermore, larger teams were able to reduce their review times more often.

The positive effect of longer targets on whether teams achieved them was observed to be minimal. This finding is supported by Parkinson's law~\cite{parkinson_cyril_parkinsons_1955}: all allocated time tends to be spent on a task, or the task adjusts to fit the time. On the other hand, some of the results failed to reflect the findings from prior research. Based on Little's law~\cite{chhajed_building_2008}, limiting open PRs should decrease the PR review time, but no statistical correlation was found. Teams with more slack\_users missed their targets more often but performed well on absolute cycle times. Larger teams, therefore, set too ambitious targets. The time the WA has been in use had a decreasing effect on whether teams met their targets, although much smaller than anticipated. 

The top-level practical implication of the study is that certain working agreements do change behavior. Moreover, the result indicates that change requires more than metrics and the importance of setting meaningful targets. Multiple connections between a newly introduced metric and a changed behavior were reported. Most notably, the value of individual Slack notifications, team-level daily reports, and the PR review time limit was observed. 

As a part of future work, a more thorough background profile for teams would provide a more complete view of the teams' situation, allowing, for example, research on what kind of WAs work for different teams. The value of DM notifications has been shown by~\citet{maddila_nudge_2022} and supported by this study. It would be interesting to see a more detailed analysis of the most efficient notifications. Furthermore, explicit ways of working that currently can not be configured in Swarmia are more challenging to track objectively but could provide valuable insights.