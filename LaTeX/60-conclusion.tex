\chapter{Conclusion}

The teams' way to use Swarmia feature called Working Agreements was studied. The way teams select and configure these ways of working, as well as the configuration's relationship with the team's software development cycle times were researched. 

A majority of teams configured four or less Working Agreements. The most popular WAs are limits for PR cycle time and PR review time. Regarding different types of issues, teams tend to gravitate towards Working Agreements that deal with stories. 

PRCT and IPCT are shown to increase when teams limit the number of open pull requests. On the other hand, discouraging direct pushes to the main branch and limiting pull request review times decreased these metrics. Regarding team characteristics, more team members using Slack notification reduced the PRCT and using Daily Digest, a team-level daily report, reduced IPCT. Furthermore, larger teams are able to reduce their review times more often.

The positive effect of longer targets to whether teams achieved them proved to be minimal. This finding is supported by Parkinson's law~\cite{parkinson_cyril_parkinsons_1955}: all allocated time tends to be spent on a task, or the task adjusts to fit to the time. On the other hand, some of the results failed to reflect the findings from prior research. Based on Little's law~\cite{chhajed_building_2008}, limiting open PRs should decrease the PR review time, but no statistical correlation was found. Teams that had more slack\_users missed their targets more often, but performed well on absolute cycle times. Larger teams therefore set too ambitious targets. Finally, the time the WA has been in use had a decreasing effect on whether teams met their targets, although much smaller than anticipated. 

The top-level practical implication of the study is that metrics do change behavior. Moreover, the result indicate that change requires more than metrics and the importance of setting meaningful targets. Multiple connections between a newly introduced metric and a changed behavior were reported. Most notably, the value of individual Slack notifications, team-level daily report and the PR review time limit were proven. 

As a part of the future work, a more thorough background profile for teams would provide a more complete view of the teams' situation, allowing for example research on what kind of WAs work for different teams. The value of DM notifications has been shown by~\citet{maddila_nudge_2022} and supported by this study. It would be interesting to see a more detailed analysis on what kind of notifications are the most efficient. Furthermore, explicit ways of working that currently can not be configured in Swarmia are harder to track objectively, but could provide valuable insights. 

\listoftodos